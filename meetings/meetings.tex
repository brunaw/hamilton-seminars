\documentclass[]{article}
\usepackage{lmodern}
\usepackage{amssymb,amsmath}
\usepackage{ifxetex,ifluatex}
\usepackage{fixltx2e} % provides \textsubscript
\ifnum 0\ifxetex 1\fi\ifluatex 1\fi=0 % if pdftex
  \usepackage[T1]{fontenc}
  \usepackage[utf8]{inputenc}
\else % if luatex or xelatex
  \ifxetex
    \usepackage{mathspec}
  \else
    \usepackage{fontspec}
  \fi
  \defaultfontfeatures{Ligatures=TeX,Scale=MatchLowercase}
\fi
% use upquote if available, for straight quotes in verbatim environments
\IfFileExists{upquote.sty}{\usepackage{upquote}}{}
% use microtype if available
\IfFileExists{microtype.sty}{%
\usepackage{microtype}
\UseMicrotypeSet[protrusion]{basicmath} % disable protrusion for tt fonts
}{}
\usepackage[margin=1in]{geometry}
\usepackage{hyperref}
\hypersetup{unicode=true,
            pdftitle={Hamilton Students Seminar Series - Meetings},
            pdfborder={0 0 0},
            breaklinks=true}
\urlstyle{same}  % don't use monospace font for urls
\usepackage{graphicx,grffile}
\makeatletter
\def\maxwidth{\ifdim\Gin@nat@width>\linewidth\linewidth\else\Gin@nat@width\fi}
\def\maxheight{\ifdim\Gin@nat@height>\textheight\textheight\else\Gin@nat@height\fi}
\makeatother
% Scale images if necessary, so that they will not overflow the page
% margins by default, and it is still possible to overwrite the defaults
% using explicit options in \includegraphics[width, height, ...]{}
\setkeys{Gin}{width=\maxwidth,height=\maxheight,keepaspectratio}
\IfFileExists{parskip.sty}{%
\usepackage{parskip}
}{% else
\setlength{\parindent}{0pt}
\setlength{\parskip}{6pt plus 2pt minus 1pt}
}
\setlength{\emergencystretch}{3em}  % prevent overfull lines
\providecommand{\tightlist}{%
  \setlength{\itemsep}{0pt}\setlength{\parskip}{0pt}}
\setcounter{secnumdepth}{0}
% Redefines (sub)paragraphs to behave more like sections
\ifx\paragraph\undefined\else
\let\oldparagraph\paragraph
\renewcommand{\paragraph}[1]{\oldparagraph{#1}\mbox{}}
\fi
\ifx\subparagraph\undefined\else
\let\oldsubparagraph\subparagraph
\renewcommand{\subparagraph}[1]{\oldsubparagraph{#1}\mbox{}}
\fi

%%% Use protect on footnotes to avoid problems with footnotes in titles
\let\rmarkdownfootnote\footnote%
\def\footnote{\protect\rmarkdownfootnote}

%%% Change title format to be more compact
\usepackage{titling}

% Create subtitle command for use in maketitle
\providecommand{\subtitle}[1]{
  \posttitle{
    \begin{center}\large#1\end{center}
    }
}

\setlength{\droptitle}{-2em}

  \title{Hamilton Students Seminar Series - Meetings}
    \pretitle{\vspace{\droptitle}\centering\huge}
  \posttitle{\par}
    \author{}
    \preauthor{}\postauthor{}
      \predate{\centering\large\emph}
  \postdate{\par}
    \date{2019-2020}


\begin{document}
\maketitle

\hypertarget{meetings}{%
\section{Meetings}\label{meetings}}

\hypertarget{st-meeting-june-4-2019}{%
\subsection{1st meeting: June 4, 2019}\label{st-meeting-june-4-2019}}

\hypertarget{summary}{%
\subsection{Summary}\label{summary}}

\begin{itemize}
\tightlist
\item
  The seminars will start at the end of September (start of term 2)
\item
  A priori, we will have one seminar every week
\item
  Abstracts should be asked to the speakers one week in advance, for us
  to advertise it and maybe make a small booklet after
\item
  The schedule needs to be discussed over Slack:

  \begin{itemize}
  \tightlist
  \item
    The order can be voluntary
  \item
    If no one volunteers to present at some date, we'll randomly assign
    a name
  \end{itemize}
\item
  Damien will check if there's any funding possibility for the seminars
  (for food, for example)
\item
  In general, the pesenter's supervisor will not be invited, unless the
  presenter decides otherwise, for her or his own reasons
\end{itemize}

\hypertarget{action-points}{%
\subsection{Action points}\label{action-points}}

\begin{itemize}
\tightlist
\item
  Organise the mailing list - Hazel {[}TO DO{]}
\item
  Set up (or check how to setup up) the webpage - Tristan {[}TO DO{]}
\item
  Make a poll to decide the best day and time for the seminars - Bruna
  {[}DONE{]}
\item
  After the day/time is decided, book the seminar room with Kate - Bruna
  {[}TO DO{]}
\end{itemize}

\hypertarget{nd-meeting-june-27-2019}{%
\subsection{2nd meeting: June 27, 2019}\label{nd-meeting-june-27-2019}}

\hypertarget{summary-1}{%
\subsection{Summary}\label{summary-1}}

\begin{itemize}
\tightlist
\item
  We set up the day and time of the week for the seminars to be every
  Thursday at 11am;
\item
  Hazel communicated that we'll have to use the current mailing list for
  advertising, and that the emails should be sent out by Kate/Rosemary;
\item
  The seminars will start in September 26;
\item
  The current schedule is available at:
\end{itemize}

\texttt{https://docs.google.com/spreadsheets/d/1BkxZIgsHCDD-ASGVtmsn6NoN7O5q8lcvXDopfqrsvBk/edit?usp=sharing}
- Apparently we can't get any money for food, but we'll try asking Ken
once more

\hypertarget{action-points-1}{%
\subsection{Action points}\label{action-points-1}}

\begin{itemize}
\tightlist
\item
  Book the seminar room with Kate - Bruna
\item
  Organise the mailing list - Hazel
\item
  Set up (or check how to setup up) the webpage - Tristan
\end{itemize}

\hypertarget{rd-meeting-29th-august}{%
\subsection{3rd meeting: 29th August}\label{rd-meeting-29th-august}}

\hypertarget{summary-2}{%
\subsection{Summary}\label{summary-2}}

\begin{itemize}
\tightlist
\item
  We agreed to move the date of the first seminar to October 1, due to
  Bruna being away on September 24
\item
  We agreed that the duration of the seminar will be flexible, given
  that some presenters might want/have more details about their research
  than words to present. The presentations will\\
  be between 30\textasciitilde{}50 minutes long, with 10 minutes of
  questions at maximum.
\item
  The last presenter should be responsible for buying the next week's
  snacks, and this money will possible be refunded later. No rule needs
  to be applied to snacks, anything is welcomed.
\item
  Damien will send out the first email to kick-off the seminars, and
  after it Hazel \& Bruna will be in charge of the mailing list (added
  with whoever else signs in)
\end{itemize}

\hypertarget{action-points-2}{%
\subsection{Action points}\label{action-points-2}}

\begin{itemize}
\tightlist
\item
  \textbf{Damien}: will check at the Hamilton Staff meeting what is out
  budget for the coffee-break (the initial idea is 30 euros for the
  first meeting, and 15 for the following ones)
\item
  \textbf{Hazel \& Bruna}: need to finish setting up the mailing list
  and promoting
\item
  \textbf{Aoife}: will be resposible for making the weekly poster
\item
  \textbf{Matt}: will be responsible for buying the first
  coffee-break/snacks
\end{itemize}


\end{document}
